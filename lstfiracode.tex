% !TeX program = XeLaTeX
% !TeX encoding = UTF-8 Unicode
%
\documentclass[11pt]{article}
\frenchspacing
\setlength\parskip{1.5pt plus 1pt minus 0.5pt}
\usepackage{fontspec}
\setmainfont{TeX Gyre Pagella}
\linespread{1.05}\selectfont
\setmonofont{Fira Code}[
  Scale=0.87096774193548387, % 459/527
  Contextuals=Alternate
]
\newfontfamily\LMMono{Latin Modern Mono}[
  Scale=1.06496519721577726  % 459/431
]
\usepackage{microtype}
\usepackage{xcolor}

\usepackage{listings}
\usepackage[verbatim]{lstfiracode}
\lstset{
  language=C++,
  style=FiraCodeStyle,
  basicstyle=\ttfamily,
  commentstyle=\color{orange}
}
\makeatletter
\lstnewenvironment{LaTeXlisting}{%
  \RestoreVerbatimBehavior
  \lst@CCPutMacro
    \lst@ProcessOther {"2D}{\lst@ttfamily{-{}}{-{}}}
    \@empty\z@\@empty
  \lstset{
    style={},
    literate={},
    language=[LaTeX]TeX,
    basicstyle=\small\LMMono,
    basewidth=0.525em,
    commentstyle=\itshape
  }%
}{}
\makeatother

\usepackage{booktabs}
\usepackage{hyperref}
\hypersetup{
  colorlinks=true
}
\usepackage{geometry}
\geometry{
  width=5.7in
}

\newcommand*\myemail{ruixizhang42@gmail.com}
\newcommand*\dash{}
\DeclareRobustCommand\dash{%
  \unskip\nobreak\thinspace\textemdash\thinspace\ignorespaces
}
\newcommand\versionhistory[3]{%
  % #1: version number, #2: date, #3: description
  \item[\ttfamily v#1]#3\hfill#2%
}
\title{The \texttt{lstfiracode} package}
\author{Ruixi~Zhang\thanks{\href{mailto:\myemail}{\nolinkurl{\myemail}}.}}
\date{xxxx/xx/xx\enskip v0.1d}

\begin{document}

\maketitle

\microtypesetup{protrusion=false}
\tableofcontents
\microtypesetup{protrusion=true}

\section{Introduction}

The Fira Code\footnote{See \url{https://github.com/tonsky/FiraCode}.} family
of fonts, created by Nikita Prokopov, is a monospaced typeface with
programming ligatures.
It is attempting for many people, me included, to use Fira Code for source
code listings.
However, the \verb|lstlisting| environment from the \verb|listings| package
does not support ligatures naively. To produce the desired output, one must
specify all necessary ligatures via the \verb|literate| key
of \verb|\lstset|, which can be tedious.

The \verb|lstfiracode| package defines a ready-to-use \verb|listings| style,
\verb|FiraCodeStyle|, which pre-specifies 125 ligatures
(note that Fira Code v1.206 has 130 ligatures in total).
You may \emph{append} the remaining 5 ligatures to the \verb|FiraCodeStyle|
literate list via a new key \verb|moreliterate|, without unintentionally
erasing all existing ligatures via \verb|literate|.

The \verb|lstfiracode| package also provides a package option,
\verb|verbatim|, along with three switches
\verb|\ActivateVerbatimLigatures|, \verb|\DeactivateVerbatimLigatures|
and \verb|\RestoreVerbatimBehavior| to support using Fira Code
in the \verb|verbatim| environment.

This package does \emph{not} provide the Fira Code font files.
The newest version of the fonts can be downloaded at
\url{https://github.com/tonsky/FiraCode/releases}.

\section{Usage}

To access \verb|FiraCodeStyle|, simply load \verb|lstfiracode| \emph{after}
\verb|listings|. Here is how you may setup your document:
\begin{LaTeXlisting}
    \documentclass{article}
    \usepackage{fontspec}
    \setmonofont{FiraCode-Regular.otf}[
      BoldFont=FiraCode-Bold.otf,
      Contextuals=Alternate  % Activate the calt feature
    ]
    \usepackage{xcolor}
    \usepackage{listings}
    \usepackage[verbatim]{lstfiracode} % Activate ligatures in verbatim
    \lstset{
      language=C++,
      style=FiraCodeStyle,   % Use predefined FiraCodeStyle
      basicstyle=\ttfamily,  % Use \ttfamily for source code listings
      commentstyle=\color{orange}
    }
    \begin{document}
    \begin{verbatim}
    A<-www>>=B
    \end{verbatim}
    \begin{lstlisting}
    /* A simple C++ program */
    int main() {
        cout << "Hello World"; // prints Hello World
        return 0;
    }
    \end{lstlisting}
    \end{document}
\end{LaTeXlisting}
which produces the following \verb|verbatim| (observe the {\LMMono<-},
the {\LMMono www} and the {\LMMono>>=} ligatures):
\begin{verbatim}
    A<-www>>=B
\end{verbatim}
and the following \verb|lstlisting| (observe the {\LMMono++}
and the {\LMMono<<} ligatures):
\begin{lstlisting}
    /* A simple C++ program */
    int main() {
        cout << "Hello World"; // prints Hello World
        return 0;
    }
\end{lstlisting}

\section{Package features}

\subsection{Package option and user commands}

The \verb|lstfiracode| package provides one package option and three user
commands, described below.

You may load the \verb|lstfiracode| package with the option \verb|verbatim|,
or equivalently \verb|verbatim=true|. This activates all Fira Code ligatures
in the \verb|verbatim| environment.
\begin{LaTeXlisting}
    % Activate Fira Code ligatures in verbatim
    \usepackage[verbatim]{lstfiracode}
    % is the same as
    \usepackage[verbatim=true]{lstfiracode}
\end{LaTeXlisting}
Loading the package without any option (the default), or equivalently with
the option \verb|verbatim=false|, \emph{does not alter} how the
\verb|verbatim| environment is handled.
\begin{LaTeXlisting}
    % Leave verbatim unaltered
    \usepackage{lstfiracode}
    % is the same as
    \usepackage[verbatim=false]{lstfiracode}
\end{LaTeXlisting}

You may change your mind in the middle of your document, so there are
three switches for such purpose:
\begin{description}
\item[\texttt{\textbackslash ActivateVerbatimLigatures}]
Activate all Fira Code ligatures in \verb|verbatim|. This is executed
automatically with the package option \verb|verbatim=true|.
\item[\texttt{\textbackslash DeactivateVerbatimLigatures}]
Suppress \emph{almost all} Fira Code ligatures in \verb|verbatim|.
Currently, it cannot break the {\LMMono\#\{} and the {\LMMono|\}} ligatures.
You should use Fira Mono if you wish to avoid ligatures altogether.
\item[\texttt{\textbackslash RestoreVerbatimBehavior}]
Restore how \verb|verbatim| is originally handled by \LaTeX.
\end{description}
These switches can overwrite each other and they act \emph{locally}.
For instance, the following three \verb|verbatim| environments
\begin{LaTeXlisting}
    \begingroup
    \ActivateVerbatimLigatures
    \begin{verbatim}
    A<-www>>=B % Fira Code ligatures activated
    \end{verbatim}
    \DeactivateVerbatimLigatures
    \begin{verbatim}
    A<-www>>=B % Fira Code ligatures deactivated
    \end{verbatim}
    \RestoreVerbatimBehavior
    \begin{verbatim}
    A<-www>>=B % Hmm...
    \end{verbatim}
    \endgroup
\end{LaTeXlisting}
produce, respectively,
\begingroup
\ActivateVerbatimLigatures
\begin{verbatim}
    A<-www>>=B % Fira Code ligatures activated
\end{verbatim}
\DeactivateVerbatimLigatures
\begin{verbatim}
    A<-www>>=B % Fira Code ligatures deactivated
\end{verbatim}
\RestoreVerbatimBehavior
\begin{verbatim}
    A<-www>>=B % Hmm...
\end{verbatim}
\endgroup

Since v0.1d, two synonyms are introduced:
\begin{description}
\item[\texttt{\textbackslash EnableVerbatimLigatures}]
Synonym for \verb|\ActivateVerbatimLigatures|.
\item[\texttt{\textbackslash DisableVerbatimLigatures}]
Synonym for \verb|\DeactivateVerbatimLigatures|.
\end{description}

\subsection{The \texttt{FiraCodeStyle} listings style}

The ligatures of Fira Code are treated as literate programming by the
\verb|lstfiracode| package. These ligatures are specified via the
\verb|literate| key in defining \verb|FiraCodeStyle|.
The definition of \verb|FiraCodeStyle| looks like this:
\begin{LaTeXlisting}
    \lstdefinestyle{FiraCodeStyle}{
      basewidth=0.6em,
      literate=
        {www}{{www}}3
        ...  % All other necessary ligatures in Fira Code
    }
\end{LaTeXlisting}

Thus, \verb|FiraCodeStyle| specifies \verb|basewidth| explicitly and
lists \emph{almost all} literate replacements.
\emph{It does not contain any font changing commands}
because users may load Fira Code according to their preferences.
In the case that \verb|\ttfamily| corresponds to Fira Code,
be sure to specify \verb|basicstyle=\ttfamily| \emph{in addition to}
\verb|style=FiraCodeStyle|, i.e.,
\begin{LaTeXlisting}
    \usepackage{listings}
    \usepackage{lstfiracode}
    % Assume that you have set Fira Code via \setmonofont
    % Then remember to also specify basicstyle
    \lstset{style=FiraCodeStyle,basicstyle=\ttfamily}
\end{LaTeXlisting}

\subsection{The missing ligatures and the new key \texttt{moreliterate}}

You may notice that some ligatures in Fira Code are still missing. Well,
there are 5 such ligatures\dash strictly speaking, they \emph{are} listed
as literate replacements in the definition of \verb|FiraCodeStyle|,
but are simply commented out:
\begin{center}
\LMMono
\begin{tabular}{c c c c c}
\toprule
\multicolumn{5}{c}{\rmfamily The ``missing'' ligatures} \\
\midrule
/* & */ & // & /// & ;; \\
\bottomrule
\end{tabular}
\end{center}
These particular combinations of characters usually indicate comment mode.
If they were to be implemented as literate replacements, they would break
how the \verb|listings| package handles comment highlighting.

Nevertheless, you can still \emph{append} these ligatures to the
\verb|FiraCodeStyle| literate list.
Say, you want to activate the {\LMMono;;} ligature in your
{\LMMono C++} code. \emph{But you cannot simply write}
{\LMMono\textbackslash lstset\{style=FiraCodeStyle,%
literate=\{;;\}\{\{;;\}\}2\}}
\emph{because this will erase all predefined ligatures, leaving only the}
{\LMMono;;} \emph{ligature}.
Instead, you should use the new key\dash
\verb|moreliterate|\dash to add more literate replacements:
\begin{LaTeXlisting}
    % Let's add more ligatures
    \lstset{
      language=C++,
      style=FiraCodeStyle,
      basicstyle=\ttfamily,
      moreliterate=
        {;;}{{;;}}2
        {///}{{///}}3
    }
\end{LaTeXlisting}

\section{Troubleshooting}

The \verb|lstfiracode| package is maintained at GitHub. Please make each
bug report with a \emph{minimal example} at
\url{https://github.com/RuixiZhang42/lstfiracode/issues}.
Pull requests are welcome.

\section*{Version history}

\begin{description}
\versionhistory{0.1d}{xxxx/xx/xx}{%
  Added synonyms
    \texttt{\textbackslash Enable}/%
    \texttt{\textbackslash DisableVerbatimLigatures}.%
}
\versionhistory{0.1c}{2018/12/24}{%
  Removed \texttt{Ligatures=Common}
    from \texttt{README.md} and \texttt{lstfiracode.tex}
    (see \href{https://github.com/RuixiZhang42/lstfiracode/issues/1}{%
    \#1}).
  Re-implemented
    \texttt{\textbackslash DeactivateVerbatimLigatures}.%
}
\versionhistory{0.1b}{2018/12/20}{%
  Updated \texttt{FiraCodeStyle} literate list.
  Added \texttt{\textbackslash RestoreVerbatimBehavior}.
  Re-implemented
    \texttt{\textbackslash Activate}/%
    \texttt{\textbackslash DeactivateVerbatimLigatures}.%
}
\versionhistory{0.1a}{2018/12/16}{%
  Initial release.%
}
\end{description}

\end{document}